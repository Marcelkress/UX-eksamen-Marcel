
\section{Introduction}

This paper will analyze the subject of difficulty in video games by conducting a qualitative data analysis on four interviews. The paper is centered around three research questions which provide the motivation for researching the topic and will aim to answer these questions based on the findings from the data analysis. 

The methodology of sampling, procedure and data analysis will be described, followed by the findings and a display of them. A discussion will describe each theme and how it can be interpreted to answer the research questions. Additionally, it will discuss ways reliability and validity of the paper could have been improved and disclose potential biases present in paper or test methodology. Lastly, the paper will conclude on whether and how the research questions can be answered successfully with the given results. 

\section{Motivation}

The motivation for this research topic comes from a game design perspective. Tailoring difficulty is an important part of game design, and this paper aims to contribute to the field of balancing difficulty. The goal of the paper is to provide relevant information through the results, that game developers might utilize to design their games. While this research most likely will not be sufficient in providing all information required to balance a game, it aims to present some initial considerations for developers to regard.

\section{Research questions} \label{Research questions}

\begin{enumerate}
    \item How do experienced players experience difficulty in video-games?
    \item How do difficulty settings impact immersion and engagement? 
    \item What motivates experienced players to persist through games they find difficult rather than quitting?
\end{enumerate}

\section{Related work}

% skriv hvor jeg har fundet artiklerne

Articles such as "\textit{An investigation of the effects of game difficulty on player enjoyment}" \cite{justin_t_alexander_investigation_2025}, explores the effects of difficulty on different player bases and their preferences. They found that players enjoy a game more if the difficulty reflects their experience playing a game, rather than their capabilities. The study discovered that players chose difficulty settings that weren't suited to their play style, but enjoyed the game more if it was tailored to their capabilities. Casual player enjoyed easier games, while experienced gamers enjoyed games matching their ability.

Another article \textit{"From objective to subjective difficulty evaluation in video games"}\cite{constant_objective_2017} found in their tests, that the feeling of progression and mastery in games helped players become more confident in their abilities. However this was not the focus of the study \cite{constant_objective_2017}, as it focused more on the player's perception of difficulty and their estimation on chance of failure. 

The study \textit{"How to Present Game Difficulty Choices? Exploring the Impact on Player Experience"} \cite{jan_d_smeddinck_how_2016} has a focus that is more aligned with this paper, but still not entirely. It explores the impact of difficulty choice on player experience with focus on the potential effects of autonomy. The study finds that embedded difficulty choices (Choices that the player makes in-game rather that in a menu), in their case, does not lead to strong effects on player engagement, motivation and enjoyment. It found that embedded difficulty choices (from menu), also do not have a significant impact on player immersion. 

% Brug deres definition for difficulty

\section{Methods}

\subsection{Participants and sampling}
For this study, purpose sampling was utilized to make sure the participants in the study would have knowledge and experience relevant to the study. Naturally, participants had to be habitual gamers, which in this study is defined as someone who plays games multiple times a week. Secondly by recruiting participants who actively seek out, or prefer challenging games, it can be somewhat ensured that the the sample group represent individuals who can provide valuable insights to the topic.

Additionally, through snowball sampling \cite{thomas_bjorner_qualitative_2015}, the first participants referred additional participants who fit the within the target group of the study. This sampling method avoids the bias of convenience sampling by including participants the researcher does not choose themselves. 

The sample size of the study is four, and they are all males of approximately the same age with one outlier. Even though gender is not a key consideration of this study, it is important to note that it could have an effect on the results, and thereby the reliability of the study. 

An overview of the participants is presented table \ref{Participant overview}.

\begin{table}[ht]
\centering
\begin{tabular}{|l|l|l|} \hline % p{10cm} wraps text in the second column
\textbf{Participant ID}& \textbf{Age} &\textbf{Gender}\\
\hline
1& 24 &M\\ \hline
2& 39 &M\\ \hline
 3&24 &M\\ \hline
 4&25 &M\\\hline
\end{tabular}
\caption{Participant overview}\label{Participant overview}
\end{table}


\subsection{Procedure}
This paper employs in-depth semi-structured interviews for collecting qualitative data. The goal of the interviews were to gain insight into the subject, of how the participants perceived aspects of difficulty in games. The semi-structured format allows for exploration of interesting themes that might occur while still relying on an interview guide \cite{thomas_bjorner_qualitative_2015}. 

The interviews were conducted with 4 participants who had gaming experience and considered themselves habitual gamers. The interview guide \ref{Interview guide} consisted of 12 questions excluding follow-up questions, the first two being introductory to make the participant start thinking about the subject, as well as obtaining first some first impressions. The next questions were direct questions to get information on the subject from different angles and asking follow-up questions that probed specific aspects of difficulty, such as frustration, challenge, or enjoyment. 
Lastly, a couple follow-up questions were asked to have the participant specify or further reason their answers. The last questions were also about player psychology in an attempt to find out whether personality traits have an impact on the topic. The data was collected through recording and transcribing of the interview.

The participants ID1 and ID2 served as pilot tests, but since no drastic changes were made to the guide and they were within the target group, their interviews have been kept and analyzed as part of the data.

It should be noted that the interviews have been firstly transcribed with the OpenAI Whisper model \cite{noauthor_openai_nodate}. They were then corrected manually and read through multiple times while listening and coding. All complete transcriptions in Danish can be found in the appendix: ID1  \ref{ID1 Transcript}, ID2 \ref{ID2 Transcript}, ID3 \ref{ID3 Transcript}, ID4 \ref{ID4 Transcript}.

\subsection{Data analysis}
To examine the data, a combination of two analysis methods will be used. The study focuses on the player's feelings about a topic in which their answers might differ based on the examples they provide, their current feelings on the topic at the time, and how they interpret the questions. Therefore, the method of traditional coding was deemed suitable as it allows for deep analysis of the words, themes, and underlying meanings of the participants. It however can also bring certain limitations which will be discussed later.

Concerns such as researcher bias, potential simplification of the participants meaning and the researcher unconscientiously leaking their own assumptions and experience into the coding, might color the results. Additionally the findings are from a small non-random sample group, which might not capture all themes that could occur from the sample population.

The meanings expressed by the participants will be used to represent the findings of the interview coding through meaning condensation. This is done as many of the participants expressed themselves and answered the questions through examples of their experiences with certain games. It however, does come with the risk of misinterpretation. This can occur through simple misunderstanding or researcher bias. The disconnect from the context of what the participant said, might also result in some lost meaning \cite{thomas_bjorner_qualitative_2015}. 

The results of the qualitative data analysis will be displayed according to some principles. Firstly the data has been categorized in its \textbf{authentic} form, meaning that no impressions or the like from the interview should influence this process. Secondly, all data that was found to fall within a category has been \textbf{included} in that category. This also means that some quotes might appear in multiple categories. 

As this paper is written by a single author, iterative data analysis is done in to increase reliability and validity. This means that the interviews have been coded twice, with approximately two weeks gap between each iteration in an attempt to combat biases.

\section{Findings}
\subsection{Pilot test}
Before presenting the results, the pilot test needs to be addressed. The tests with participant ID1 and ID2  were pilot tests. They served as a way to validate and make sure that the interview guide did not need further changes and captured sufficient information. The tests proved successful as the participants provided valuable insights, proving the guide's ability to capture the desired information, and were therefore kept as part of the qualitative data.

\subsection{Themes} \label{Themes}
The themes will be presented as tables of supporting quotes from the participants. Meaning condensation of the quotes in each theme will then be displayed, followed by a short discussion.

% Ikke alle participants kommer ind under hvert tema - Husk at nævn

\subsection{Theme 1: Overcoming a frustrating and difficult challenge is fun}\label{Theme 1}
% Table start
\begin{table}[H]

\centering
\begin{tabular}{@{}c p{11cm}@{}}
\toprule
\textbf{Participant ID} & \textbf{Theme 1} (Translated from Danish)\\
\midrule

1 & \textit{"Overcoming challenges and getting better is the best reward."}

\vspace{0.3cm}

\textit{"And it feels so nice to... After having been frustrated, when you return to it, and do really well."}\\

\midrule

3& \textit{"… But I think that it comes down to that I think it's fun to overcome this challenge, right"}

\vspace{0.3cm}

\textit{"There is something nice about struggling with something and then suddenly it clicks."}

\vspace{0.3cm}

\textit{"What got me was primarily that i wanted to get through Path of Pain[specific challenge in a game]. It was that I knew there was this super challenge with difficult platforming. You didn't get anything for it, it was just difficult.[...] And that can motivate me. Just the fact that it's difficult. I think that's awesome"}\\

\midrule

4& \textit{"But of course I'm also motivated by the fact that it was fun. So you play this level 5 times, and its like okay this time I will get through the whole level without anyone noticing me, and that's awesome when you do it at last"}

\vspace{0.3cm}

\textit{"... And it doesn't feel the best because he's strong [the boss]. So you're like okay. Then you fight him a couple times and suddenly you start to learn how he moves. And then suddenly you get him under half his health. And then I'm like okay next time I got this [...] and then I come back and beat him, and that is a great feeling."}\\

\bottomrule
\end{tabular}

\caption{Quotes related to Theme 1: Overcoming a frustrating and difficult challenge is fun}
\label{tab:theme1-quotes}
\end{table}
% Table end

The quotes \ref{tab:theme1-quotes} of the participants can be condensed into the following:
\begin{center}
    \textbf{Overcoming a frustrating and difficult challenge is fun and a gratifying experience.}
\end{center}

This theme occurred from the participants expressing that overcoming a challenge in a game felt rewarding. This either occurred in the context that it felt like a greater reward than literal in-game rewards, as asked about in the interview questions \ref{Interview guide}, or when asked about whether they enjoy challenges in games. It can be argued that this partly answers research question 1, since players reported that they enjoy overcoming challenges in games. It suggests that they experience difficulty as a positive trait of a game. However, no participants expressed this directly, implying that the positive experience with difficulty, isn't necessarily due to a correlation between the two. 

This theme can also relate itself to research question 3. The quotes strongly suggest that the rewards upon overcoming a challenge, play an important part in motivating them. Participant 3 mentions \textit{"There is something nice about struggling with something and then suddenly it clicks."}  \ref{tab:theme1-quotes},  which reveals that the process of overcoming a challenge is also a part of the reward.

\subsection{Theme 2: Games becomes frustrating when it feels out of control and as the fault of the game}\label{Theme 2}
% Table start
\begin{table}[H]

\centering
\begin{tabular}{@{}c p{11cm}@{}}
\toprule
\textbf{Participant ID} & \textbf{Theme 2} (Translated from Danish)\\
\midrule

1 & \textit{"There are some games where you become angry at the game and you feel like its the games fault"}\\

\midrule

2& \textit{"It's not necessarily the games resistance which has made me frustrated. It can be, as a peer review designer, or just as someone experiencing an experience, that I just didn't find the game fun. That can also be a form of frustration."}\\

\midrule

3& \textit{"I think that there is a difference between frustration and the other thing. But not to say that you can't get frustrated from not beating a boss. But I definetely think that it[frustration] comes from when it feels like its out of your control"}\\

\midrule

4 & \textit{"Is it a bug that caused you to not get further? Thats not difficult thats just being stuck. Then you have to find an old save file thats an hour back and have to go through everything again. That's just a bad experience."}

\vspace{0.3cm}

\textit{"Difficulty I tink depends more on your own skill. Where frustration, I believe, has most often appeared when it feels like its out of my hands."}\\

\bottomrule
\end{tabular}

\caption{Quotes related to Theme 2: Games becomes frustrating when it feels out of control and as fault of the game}
\label{tab:theme2-quotes}
\end{table}
% Table end

The quotes \ref{tab:theme2-quotes} of the participants can be condensed into the following:
\begin{center}
    \textbf{Players find that a game or challenge becomes frustrating when avoiding penalties feels out of their control and as a fault of the game rather than themselves.}
\end{center}

This theme describes the line players drew between difficulty and frustration. While not directly addressing the research questions, this theme provides insight into how players define difficulty. As shown in the quotes this was one of the themes that occurred across all interviews, suggesting this is common among the research population. However it can't be entirely concluded upon due to the small sample size. 

\subsection{Theme 3: Having a sense of progression is motivating}\label{Theme 3}

% Table start
\begin{table}[H]

\centering
\begin{tabular}{@{}c p{11cm}@{}}
\toprule
\textbf{Participant ID} & \textbf{Theme 3} (Translated from Danish)\\
\midrule

1 & \textit{"... so that [a difficult challenge] was a bit of a roadblock, where in some way... I didn't want to play, but I still found it fun to progress even though it was tough."}\\

\midrule

2 & \textit{"When there are active changes in game play, that can be super nice. It can give a fresh take. But if it[the reward] is a thing that does nothing else than make you look at little different, as fun as it can be, like now you get a crown or something, that can be its own fun. But in relation to the game, variation in game play - that I can get behind"}\\

\midrule

3& \textit{"... but there was always this feeling of even if you were on the same boss, there was still a feeling of progression. So you could still see the light at the end of the tunnel."}

\vspace{0.3cm}

\textit{"... and you put in some time and could feel the progress. You always got a little further. You learned a little more. And all that came before was easier. That's like... That's really nice."}\\

\midrule

4 & \textit{"That's it, that's almost what makes the game good, one can say. It's this sense of progression right."}\\

\bottomrule
\end{tabular}

\caption{Quotes related to Theme 2: Having a sense of progression is motivating}
\label{tab:theme3-quotes}
\end{table}
% Table end

The quotes \ref{tab:theme3-quotes} of the participants can be condensed into the following:
\begin{center}
    \textbf{Players find a sense of progression motivating and important when facing challenges.}
\end{center}

This theme provides valuable insight into research question 3 as it directly ties into what exactly motivates players when they are facing a challenge. The quotes state that progression is enjoyable, from which it can be derived that a sense of progression plays an important part in helping players push through a challenge. The quotes however don't explicitly say that progression is something that keeps them from quitting a game, but seems at least to be related to it.

\subsection{Theme 4: Fixed difficulty is preferred}\label{Theme 4}
% Table start
\begin{table}[H]

\centering
\begin{tabular}{@{}c p{11cm}@{}}
\toprule
\textbf{Participant ID} & \textbf{Theme 4} (Translated from Danish)\\
\midrule

1 & \textit{"I think I prefer fixed because I think that it's pretty appropriate for the experience. Those who develop the games have a pretty concrete intention with what they are developing. If you want an engaging experience in a specific way, then it I guess it requires that you aim at that. So if it's meant to be difficult, then nail it in the way that it's supposed to be difficult. Because it would not be the same experience if you could easily beat it."}\\

\midrule

2& \textit{"... when i meet a game with five difficulties, it gets difficult in itself to choose. Hmm... what do I want? There is already a lot you dont know about the game already, so how could I know which one of the five difficulty settings I want. So generally I'd prefer that there only exists one."}

\vspace{0.3cm}

\textit{"Yes, most of the time there can be some sort of... That I have a preference for this Dark Souls-ish, what you see is what you get. Yea and you can kind of learn the game."}\\

\midrule

3& \textit{"That's the thing about making a decision about choosing a difficulty. It's really like a.. Ah I really want to, I want to experience the story and get through the game and all these things. Because that's the primary thing. But, it could have been better with the other thing [Fixed difficulty] right."}\\

\midrule

4 & \textit{"If we're talking from a pure immersion perspective, then I'd prefer fixed difficulty. Because again.. There isn't this possibilty of just making the game easier. Sometimes it's like you dont even need to change difficulty, but you're just like, **** it, I just want to get through, and then you do it. And then you completely break immersion by it. So with fixed you're just forced to experience the game and how hard it is, and how the game actually is. Thats also pretty cool because it's easier to balance the game."}\\

\bottomrule
\end{tabular}

\caption{Quotes related to Theme 4: Fixed difficulty is preferred}
\label{tab:theme4-quotes}
\end{table}
% Table end

The quotes \ref{tab:theme4-quotes} of the participants can be condensed into the following:
\begin{center}
    \textbf{"Players prefer fixed difficulty settings because they believe it provides a more consistent game play experience, reflecting the developer's intended challenge and preserving immersion."}
\end{center}

This theme primarily relates itself to research question 2 about difficulty options. Players expressed that it can be difficult even in itself to choose a difficulty option upon starting a game. Additionally, it seems, players from the sample group, are aware of their choice's impact on their experience, in regards to the developers choices about different settings. Meaning that they are aware that they might not get the intended experience, if they change the difficulty setting in a game. Or rather, the quotes suggest that they wish to get the intended experience, which they believe is more likely with fixed difficulty. 

\subsection{Additional observations}
The following observations aren't labeled as themes because fewer participants and phrases were found to fit them. They are included as they provide interesting points of discussion to the research questions.

They will be displayed in the same manner as the themes \ref{Themes}.

\subsection{Observation 1: When games becomes too difficult}\label{Observation 1}
% Table start
\begin{table}[H]

\centering
\begin{tabular}{@{}c p{11cm}@{}}
\toprule
\textbf{Participant ID} & \textbf{Observation 1} (Translated from Danish)\\
\midrule

1 & \textit{"I stopped because it started to feel like... Feel more like an assignment, or homework or work. More than to have a good time. And you play to have a good time and have fun. And like, as soon as it begun to get so challenging that it wasn't nice or fun anymore, then it was like doing homework. Then it was just a principle. And if it feels like that for long enough, then it dies."}\\

\midrule

2& \textit{"If I feel that it becomes too difficult, where I sit and try for a couple hours to beat it. And then it eventually it becomes... It just starts to feel like labour"}\\
\bottomrule
\end{tabular}

\caption{Quotes related to Observation 1: When games become too difficult}
\label{tab:obs1-quotes}
\end{table}
% Table end

The quotes of the participants can be condensed into the following:
\begin{center}
    \textbf{A game becomes too difficult when it starts to feel like work.}
\end{center}

This observation provides some context as to when players stop trying to beat a challenge. The lack of motivation results in the game feeling like work. It provides a point of view suggesting that the difference between a work challenge and a game challenge, is the motivation that the game provides. And as theme 3 \ref{Theme 3} established, motivation comes from progression, which can be thought of as a type of reward. Ultimately this proposes that when games stop rewarding the player, it starts to feel like labour. 

\subsection{Observation 2: Avoiding frustration by providing something else to do [in the game]}\label{Observation 2}
% Table start
\begin{table}[H]

\centering
\begin{tabular}{@{}c p{11cm}@{}}
\toprule
\textbf{Participant ID} & \textbf{Observation 2} (Translated from Danish)\\
\midrule

1 & \textit{"If you have possibilities and aren't forced to do one thing, or something repetitive, or something that doesn't feel like it makes sense. If you get the possibility to do a range of things, have options, thats definitely less frustrating because you decide what to do in the game"}\\

\midrule

4& \textit{"... I'm not high enough leveled. And then you go out and level and explore, then come back and beat the boss, thats a great feeling"}\\
\bottomrule
\end{tabular}

\caption{Quotes related to Observation 2: Avoiding frustration by providing something else to do}
\label{tab:obs2-quotes}
\end{table}
% Table end

The quotes of the participants can be condensed into the following:
\begin{center}
    \textbf{Having something else to do [in the game] is important to avoid frustration}
\end{center}

This observation also relates itself to research question 3 \ref{Research questions}. While theme 3 \ref{Theme 3} is about the general sense of progression and describes the benefits of it, this observation describes how the option to do something else(in the game), mitigates frustration. Essentially they can be viewed as describing the same concept; avoiding frustration, but from two different perspectives. Theme 3 enhances a positive concept, while this observation negates the opposite (negative) concept. This observation could also have been coded as a sub-category of theme 3. 

\section{Review: Findings, Methods, Validity and Reliability}
This section will discuss the results as a whole, including how they can be used to answer the research questions. The validity and reliability of the paper will be discussed, as well as methods as to how they could have been improved. 

\subsection{Discussion of findings}

The findings overall address the research questions to different extents. Theme 1 \textit{"Overcoming a frustrating and difficult challenge is fun"} \ref{Theme 1} and theme 2 \textit{"Games become frustrating when it feels out control and as the fault of the game"} \ref{Theme 2} relate to research question 1 \textbf{"How do Experienced gamers react to and experience difficulty in video-games?"}. The themes highlight the subjective nature of difficulty perception. What one player might view as a challenge worth overcoming, another player might view as being out of their control and frustrating showing the ambiguity of the subject. The themes show that there is no objective line to draw between when a challenge is fun, or when it becomes too difficult. Participant ID 3 stated: 
\begin{quote}
    \textit{"But I definitely think that it [frustration] comes from when it feels like its out of your control"} \ref{tab:theme2-quotes},
\end{quote}
and in slight contrast, ID 1 stated:
\begin{quote}
    \textit{"Overcoming challenges and getting better is the best reward."} \ref{tab:theme1-quotes},
\end{quote}
suggesting that there is no definite answer, at least not to be derived from the results in this paper. 

Research question 2 \textbf{"How do difficulty settings impact immersion and engagement?"} is addressed by theme 4 \textit{"Fixed difficulty preferred"} \ref{Theme 4}. It suggests that the choice of difficulty can disrupt immersion by placing a burden of choice upon the player. Participant ID 2 stated:
\begin{quote}
    \textit{"... when i meet a game with five difficulties, it gets difficult in itself to choose. Hmm... what do I want? There is already a lot you don't know about the game already, so how could I know which one of the five difficulty settings I want. "} \ref{tab:theme4-quotes},
\end{quote}
This burden of choice is also present throughout a game and arises when a challenge presents itself as the player can consider lowering the difficulty, like participant ID 4 describes:
\begin{quote}
    \textit{"Sometimes it's like you don't even need to change difficulty, but you're just like, fuck it, I just want to get through, and then you do it. And then you completely break immersion by it."} \ref{tab:theme4-quotes}
\end{quote}
Somewhat contradictorily, the study \textit{"How to Present Game Difficulty Choices? Exploring the Impact on Player Experience"} \cite{jan_d_smeddinck_how_2016} found that difficulty settings do not have an effect on player immersion. A reason for the contradicting findings could be that in this study, the participants do self reporting, which might make them more likely to say it breaks their own immersion. The idea that it's difficult to recognize one's own immersion without breaking immersion raises questions about the validity of using self-reporting methods to assess immersion. why?

Lastly, question 3 \textbf{"What motivates experienced players to persist through games they find difficult rather than quitting?"} is supported by theme 3 \textit{"Having a sense of progression is motivating"} \ref{Theme 3} and the two observations \ref{Observation 1}, \ref{Observation 2}. Theme 3 almost directly answers the question by stating that a sense of progression within the game is motivating.

Both of the observations could also serve as sub-categories of theme 3, as they describe the same overall principle of motivation. These findings together suggest that perceived difficulty depends on the players ability to progress and manage frustration through the games systems. 

\subsection{Methods, Reliability and validity}

The methods utilized for this study were not optimal in terms of getting a sample group that was representative of the target population. More participants from random as well as snowball sampling, would have resulted in more diverse and convincing themes. This would also increase the reliability and validity of this study.

When conducting the interviews, brief contributions from the researcher triggered the participants to elaborate on their opinions and responses. This might have introduced biases, but provided new insights. Focus group interviews could also have been a viable method to gather data. 

% Der var ikke mange contradicting udsagn fra participants
% Dvs. temaerne blev ikke modsagt og derved ikke falsificeret
% The themes werent falsified or contracted 

Inter-coder reliability could also have been utilized to increase validity.

\clearpage

\section{Conclusion}
The study revealed four main themes; \textit{"Overcoming a frustrating and difficult challenge is fun"} \ref{Theme 1}, \textit{"Games become frustrating when it feels out of their control"} \ref{Theme 2}, \textit{"Having a sense of progression is motivating"} \ref{Theme 3}, and \textit{"Fixed difficulty is preferred"} \ref{Theme 4}.

The motivation for the study came from a game design perspective, providing initial considerations which game developers could integrate into their projects. As this study included four participants, the findings are somewhat limited in their validity and reliability with room for improvement, but still supply interesting points of discussion and observations and future research.

Regarding research question 1; the findings show the ambiguity of how experienced players experience difficulty in video games. Themes one \ref{Theme 1} and two \ref{Theme 2} highlight this as participants describe difficulty as capable of both being a fun challenge to overcome, and as a source of frustration. Research question 2 is addressed by the findings from theme four \ref{Theme 4}, suggesting that players from the sample prefer fixed difficulty because of the consistent game play experience. The theme suggests that players are aware of their impact on their experience when changing difficulty, and wish to get the intended experience from the developers. Lastly, addressing research questions 3; the findings in theme three \ref{Theme 3} show that the players are primarily motivated by the feeling of progression within a game. Observations one \ref{Observation 1} and two \ref{Observation 2} also support this by stating that games become too difficult when they start to feel like labour, from a lack of progression, and that having something else to do in the game is important to avoid frustration.

It is important to critically asses the conclusions that can be drawn from the findings of four participants. There is too high of an impact from any one individual to skew the conclusions towards different possibilities. The sample consisted primarily of males in their mid-twenties, with a single outlier, which may also limit the diversity of the perspectives present in the sample.  However, no statements were elicited that directly contradicted the themes described, indicating a certain degree of validity and reliability in the results.

Further research is needed to understand how findings might differ across player profiles and higher sample size. For future research a study with a more varied sample in terms of age and gender, larger sample size and the method of focus group interviews is recommended to avoid the biases present in this paper and produce findings that are more convincing and varied.

In conclusion, the study revealed aspects of the complex subject of difficulty in games, and additional research is required to further understand players experience with difficulty in games.

\clearpage